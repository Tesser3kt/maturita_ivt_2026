%!TeX program = xelatex
\documentclass[aspectratio=169,xcolor=dvipsnames, t]{beamer}
\usepackage{fontspec} % Allows using custom font. MUST be before loading the theme!
\usetheme{SimplePlusAIC}
\usepackage{hyperref}
\usepackage{graphicx} % Allows including images
\usepackage{booktabs} % Allows the use of \toprule, \midrule and  \bottomrule in tables
\usepackage{svg} %allows using svg figures
\usepackage{tikz}
\usepackage{makecell}
\usepackage{wrapfig}
\usepackage[czech]{babel}

\newcommand{\R}{\mathbb{R}}
\newcommand{\N}{\mathbb{N}}
\usepackage{tikz-among-us}

\title[]{Program a programování a utrpení} % The short title appears at the bottom of every slide, the full title is only on the title page
\subtitle{Neumím programovat a Jáchym taky ne}

\author[Dusart]{Eric a Kuba}
\institute[GEVO]{Gymnázium Evolution Jižní Město}
\date{\today}


\begin{document}

\maketitlepage
{
\setbeamertemplate{background}
{
    \includegraphics[width=\paperwidth,height=\paperheight]{AICStyleData/logos/mene_polygonu_bg.png}
}
\begin{frame}[t]{Obsah}
    \tableofcontents
\end{frame}
}


\section{Počítačový program}
{
\setbeamertemplate{background}
{
    \includegraphics[width=\paperwidth,height=\paperheight]{AICStyleData/logos/mene_polygonu_bg.png}
}
\begin{frame}{Počítačový program \\(aspoň trochu do hloubky (jako jáchymovy špagety))}
    \begin{itemize}
        \item Počítačový program je sada nebo posloupnost instrukcí napsaných ve specifickém programovacím jazyku (\uv{řeč počítače v lidském stylu}) určených pro počítač (on čte ty instrukce), který tento program vykoná.   
        \item Počítačový program musí (nebo chceme aby byl) být \textbf{přesný a jednoznačný}.
        \item Tak se liší od lidské řeči, kde si příjímatelé (lidi) mohou domýšlet věci, to ale počítač neumí. 
    \end{itemize}
\end{frame}

\begin{frame}{Syntax -- Sémantika}
    \textbf{Syntax -- Sémantika}
    \begin{itemize}
        \item Syntax popisuje pravidla zápisu programu (gramatika jazyka).
        \item Sémantika definuje význam kódu. - logistika a tak\ldots
    \end{itemize}
\end{frame}

\section{Programovací jazyky}
\begin{frame}{Příklady programovacích jazyků}
    \begin{itemize}
        \item Zotročil jsem AI: \href{https://chatgpt.com/share/67c75c2c-c014-800b-a533-57120d2b885b}{link}
    \end{itemize}
\end{frame}

\subsection{Proceduranální vs. funkcioanální jazyky}
\begin{frame}{Proceduranální vs. funkcioanální jazyky}
    \begin{columns}
        \begin{column}{0.5\textwidth}
            \textbf{Proceduální -- Python, Rust, C, Java, \ldots}
            \begin{itemize}
                \item Program je sekvence příkazů a funkcí.
                \item Jsou běžnější v praxi.
            \end{itemize}
        \end{column}
        \begin{column}{0.5\textwidth}
            \textbf{Funkcionální -- Haskell, F\#, Scala, OCaml, Idris, Lean, \ldots}
            \begin{itemize}
                \item Zaměřují se na to, co se má vypočítat, nikoli na to, jak se to má provést.
                \item Často se používá rekurze
                \item V ideálním funkcionálním programování jsou data neměnná. To znamená, že po vytvoření hodnoty ji nelze změnit.
                \item Místo změny dat se vytvářejí nové hodnoty.
            \end{itemize}
        \end{column}
    \end{columns}
\end{frame}





\subsection{Způob překladu}
\begin{frame}{Způob překladu}
    \begin{columns}
        \begin{column}{0.5\textwidth}
            \textbf{Kompilované} -- C, Java, Cobol
            \begin{itemize}
                \item Před spuštěním se strčí do zázračného stroje, který to přeloží do strojového kódu
                \item 
            \end{itemize}
        \end{column}
        \begin{column}{0.5\textwidth}
            \textbf{Interpretované} -- Python, BASIC, Shell
            \begin{itemize}
                \item Překládá za letu $\Rightarrow$ je pomalejší
                \item 
            \end{itemize}
        \end{column}
    \end{columns}
\end{frame}

\begin{frame}{Source code, assembly, low-level a \\ high-level jazyky}
    \begin{itemize}
        \item Source code – kód napsaný programátorem v čitelném jazyce (např. C, Python).
\item Assembly – nízkoúrovňový jazyk blízký strojovému kódu, ale stále čitelnější pro člověka.
\item Low-level jazyky – blízko hardware, např. C, Rust. Používají ukazatele (pointers), což jsou adresy v paměti, které umožňují přímou manipulaci s daty.
\item High-level jazyky – blíže lidské logice, např. Python, Java. Mají automatickou správu paměti (garbage collector), který čistí nevyužitou paměť, aby programátoři nemuseli ručně spravovat alokaci a dealokaci.
\end{itemize}
\end{frame}




\section{Bonus: Recepty na špagety}
\begin{frame}{Bonus}
    amongus (nepodarili se zmensit :o)

    \begin{minipage}{0.25\textwidth}
        \centering
    \begin{tikzpicture}
        \amongUsIII{red}{blue}
        \amongUsIII[xscale=-1,shift={(-12,0)}]{yellow}{blue}
        \amongUsIII[shift={(12,0)},scale=0.5]{violet}{orange}
        \amongUsIII[scale=-0.5,shift={(-30,-16)}]{blue}{green}
        \end{tikzpicture}
    \end{minipage}
\end{frame}
\begin{frame}{Recept na špagety Bolognese - 1}
    \begin{columns}
        \begin{column}{0.5\textwidth}
            \textbf{Ingredience:}
            \tiny{
    \begin{itemize}
        \item 400 g mletého hovězího masa
        \item 1 cibule
        \item 2 stroužky česneku
        \item 1 mrkev
        \item 2 plechovky rajčatového protlaku
        \item 200 ml červeného vína
        \item 400 g špaget
        \item Olivový olej
        \item Sůl, pepř, oregano, bazalka
        \item Parmazán na posypání
    \end{itemize}}
        \end{column}
        \begin{column}{0.5\textwidth}
    \textbf{Postup:}
    \tiny{
    \begin{enumerate}
        \item Na pánvi rozehřejte olivový olej a osmahněte na něm nadrobno nakrájenou cibuli a česnek.
        \item Přidejte nastrouhanou mrkev a mleté maso. Orestujte, dokud maso nezhnědne.
        \item Přilijte červené víno a nechte ho odpařit.
        \item Přidejte rajčatový protlak, osolte, opepřete a přidejte oregano a bazalku. Nechte dusit na mírném ohni asi 30 minut.
        \item Mezitím uvařte špagety podle návodu na obalu.
        \item Podávejte špagety s omáčkou a posypané strouhaným parmezánem.
    \end{enumerate}}
        \end{column}
    \end{columns}
    
\end{frame}

\begin{frame}{Recept na špagety Bolognese - 2}
    \href{https://github.com/kubasim}{Kubův github (link), ale má ty recepty private (ty špagety jsou až moc dobré, aby si mohl dovolit sdílet recept (dostal by všechny Michelinské hvězdy a stal by se milionářem (život by pak byl moc jednoduchej))).}
\end{frame}
\begin{frame}{Recept na špagety Bolognese - 3}
    \begin{figure}
        \centering
         \includegraphics[width=0.30\textwidth]{obrazek}
        \caption{Eric} 
    \end{figure}
\end{frame}
\begin{frame}{Video co vše vysvětluje}
    \begin{figure}
        \href{https://www.youtube.com/watch?v=2lVDktWK-pc&pp=ygUhcHJvZ3JhbW1pbmcgbGFuZ3VhZ2VzIGZvciBkdW1taWVz}{link}
    \end{figure}
\end{frame}
}
\finalpagetext{Děkujeme za pozornost}
\makefinalpage
\end{document}
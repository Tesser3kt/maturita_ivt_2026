%!TeX program = xelatex
\documentclass[aspectratio=169,xcolor=dvipsnames, t]{beamer}
\usepackage{fontspec} % Allows using custom font. MUST be before loading the theme!
\usetheme{SimplePlusAIC}
\usepackage{hyperref}
\usepackage{graphicx} % Allows including images
\usepackage{booktabs} % Allows the use of \toprule, \midrule and  \bottomrule in tables
\usepackage{svg} %allows using svg figures
\usepackage{tikz}
\usepackage{makecell}
\usepackage{wrapfig}
\usepackage[czech]{babel}

\newcommand{\R}{\mathbb{R}}
\newcommand{\N}{\mathbb{N}}
\usepackage{tikz-among-us}

\title[]{Program a programování a utrpení} % The short title appears at the bottom of every slide, the full title is only on the title page
\subtitle{Neumím programovat a Jáchym taky ne}

\author[Dusart]{Eric a Kuba}
\institute[GEVO]{Gymnázium Evolution Jižní Město}
\date{\today}


\begin{document}

\maketitlepage
{
\setbeamertemplate{background}
{
    \includegraphics[width=\paperwidth,height=\paperheight]{AICStyleData/logos/mene_polygonu_bg.png}
}
\begin{frame}[t]{Obsah}
    \tableofcontents
\end{frame}
}


\section{Počítačový program}
{
\setbeamertemplate{background}
{
    \includegraphics[width=\paperwidth,height=\paperheight]{AICStyleData/logos/mene_polygonu_bg.png}
}
\begin{frame}{Počítačový program (aspoň trochu do hloubky (jako jáchymovy špagety))}
    \begin{itemize}
        \item Počítačový program je sada nebo posloupnost instrukcí napsaných ve specifickém programovacím jazyku (\uv{řeč počítače v lidském stylu}) určených pro počítač (on čte ty instrukce), který tento program vykoná.   
        \item  
    \end{itemize}
\end{frame}

\section{Programovací jazyky}
\begin{frame}
    
\end{frame}

}


\finalpagetext{Děkuji za pozornost}
%----------------------------------------------------------------------------------------
\makefinalpage
%----------------------------------------------------------------------------------------
\end{document}

\documentclass[a4paper,12pt,answers]{exam}

\usepackage[english,czech]{babel}
% Fonts %
\usepackage{fouriernc}
\usepackage[T1]{fontenc}

% Colors %
\usepackage[dvipsnames]{color}
\usepackage{xcolor}

% Page Layout %
\usepackage[margin=1in]{geometry}

\pagestyle{headandfoot}
\runningfooter{}{\thepage}{}

% Math stuff %
\usepackage{amsmath}

% Fancy Headers %
\usepackage{enumitem}

% Tables %
\usepackage{booktabs}
\usepackage{tabularx}

\newcommand{\clr}{\textcolor{BrickRed}}
\newcommand{\clb}{\textcolor{RoyalBlue}}
\newcommand{\clg}{\textcolor{ForestGreen}}
\newcommand{\clf}{\textcolor{Fuchsia}}

% Rename solution
\renewcommand{\solutiontitle}{\noindent\textbf{Řešení:}\enspace}

% Document %
\begin{document}

\section*{Vzorová zadání lehkých úloh na číselné soustavy}

\begin{questions}
 \question Sečtěte čísla $\mathtt{33212_8}$ a $\mathtt{65127_8}$.
 \begin{solution}
  Sčítání pod sebou to jistí. Jenom je třeba dávat pozor na to, že $\mathtt{7}$
  je poslední číslice osmičkové soustavy. Po ní následuje $\mathtt{10}$.
  Dostaneme
  \[
   \begin{array}{rr}
    & \mathtt{33212} \\
    +& \mathtt{65127} \\
    \hline
     & \mathtt{120341}
   \end{array},
  \]
  protože třeba $\mathtt{7_8 + 2_8 = 11_8}$ nebo $\mathtt{3_8 + 5_8 = 10_8}$.
 \end{solution}

 \question Převeďte číslo $\mathtt{6EF_{16}}$ do desítkové soustavy.
 \begin{solution}
  Rozložíme si číslo na mocniny desítky:
  \[
   \mathtt{6EF = 6 \cdot 10_{16}^2 + E \cdot 10_{16}^{1} + F \cdot 10_{16}^{0}}.
  \]
  V desítkové soustavě je $\mathtt{6_{16} = 6_{10}}$, $\mathtt{E_{16} =
  14_{10}}$, $\mathtt{F_{16} = 15_{10}}$ a $\mathtt{10_{16} = 16_{10}}$. Takže
  počítáme
  \begin{align*}
   \mathtt{6EF}&~\mathtt{=\,6 \cdot 10_{16}^2 + E \cdot 10_{16}^{1} + F \cdot
   10_{16}^{0} = 6 \cdot 16^2 + 14 \cdot 16^{1} + 15 \cdot 16^{0}}\\
               &~\mathtt{=\,1775}.
  \end{align*}
  Proto $\mathtt{6EF_{16} = 1775_{10}}$.
 \end{solution}

 \question Převeďte číslo $\mathtt{435}$ do dvojkové soustavy.
 \begin{solution}
  Budeme $\mathtt{435}$ dělit se zbytkem $\mathtt{2}$, dokud se nedostaneme na
  $\mathtt{0}$. Posloupnost zbytků je pak hledaným číslem v dvojkové soustavě.
  Postupně spočteme
  \begin{center}
  {\def\arraystretch{0.9}
   \begin{tabular}{
    >{\centering\arraybackslash}p{1.5cm}|
    >{\centering\arraybackslash}p{1.5cm}
   }
    \textbf{Podíl} & \textbf{Zbytek}\\
    \toprule
    \texttt{217} & \texttt{1}\\
    \midrule
    \texttt{108} & \texttt{1}\\
    \midrule
    \texttt{54} & \texttt{0}\\
    \midrule
    \texttt{27} & \texttt{0}\\
    \midrule
    \texttt{13} & \texttt{1}\\
    \midrule
    \texttt{6} & \texttt{1}\\
    \midrule
    \texttt{3} & \texttt{0}\\
    \midrule
    \texttt{1} & \texttt{1}\\
    \midrule
    \texttt{0} & \texttt{1}
   \end{tabular}
  }
  \end{center}
  Takže $\mathtt{435_{10}} = \mathtt{110110011_2}$.
 \end{solution}
\end{questions}

\end{document}

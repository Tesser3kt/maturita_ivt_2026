\documentclass{beamer}


\usetheme{Madrid}
\usepackage{graphicx} % Allows including images
\usepackage{booktabs} % Allows the use of \toprule, \midrule and \bottomrule in tables
\usepackage[czech]{babel}

%----------------------------------------------------------------------------------------
%	TITLE PAGE
%----------------------------------------------------------------------------------------

\title[počítačových lol]{Topologie sítí} % The short title appears at the bottom of every slide, the full title is only on the title page

\author{Jáchym Löwenhöffer} % Your name
\institute[GEVO] % Your institution as it will appear on the bottom of every slide, may be shorthand to save space
{
Gravitaci Emulující Vágní Opar \\ % Your institution for the title page
\medskip
\textit{jachym.lowenhoffer@gmail.com} % Your email address
}
\date{\today} % Date, can be changed to a custom date

\graphicspath{{./pics/}}

\AtBeginSection[]
{
  \begin{frame}
  \vfill
  \centering
  \begin{beamercolorbox}[sep=8pt,center,shadow=true,rounded=true]{title}
    \usebeamerfont{title}\insertsectionhead\par%
  \end{beamercolorbox}
  \vfill
  \end{frame}
}

\begin{document}

\begin{frame}
	\titlepage % Print the title page as the first slide
\end{frame}

\begin{frame}
	\frametitle{Přehled} % Table of contents slide, comment this block out to remove it
	\tableofcontents % Throughout your presentation, if you choose to use \section{} and \subsection{} commands, these will automatically be printed on this slide as an overview of your presentation
\end{frame}

\begin{frame}{Co je to sakra síť?}
 Když už máme jeden funkční počítač tak, by mu ale samotnému bylo smutno. Proto
 si se svými kamarády chce povídat. Povídá si s nimi přes fyzický kabel, nebo online -
 fyzická, logická síť.

 Aby si počítače rozuměli tak potřebují nějaký jazyk - o tom bude mluvit maják.
 Většinou jsou specifické pro určité topologie. My budeme uvažovat jen že si
 posílají balíčky informací - packety.
\vfill

Důležité je u sítí předpokládat, že co se pokazit může, tak to se pokazí. I věci
které bereme implicitně jako bezchybné, se v praxi rozbíjejí a způsobují výpadky.
Jestliže si je nemůžeme dovolit tak musíme zavést nadbytečné součástky, které zaskočí ve
chvíli kdy se něco pokazí.

 
\end{frame}


\section{Metriky}
\label{sec:metriky}

\begin{frame}{Co je dobrá síť?}
	Taková, že z ní chycená ryba nevyskočí a zároveň skrz ní dokáže protéct voda
	čímž šetří palivo ryboloveckých lodí.
	\vfill
	\centering
	\Large
	Taková, které splňuje naše požadavky.

	\normalsize
	\raggedright
	\vfill
	Každá má své pozitiva a negativa je na nás abychom je balancovali. Jsou určitá
	kritéria která od sítě můžeme chtít. Ty se odvíjejí od toho jak síť stavíme,
	jak se o ní staráme nebo jak je bezpečná.
\end{frame}

\begin{frame}{Kritéria}
 Na základě těchto kritérií budeme hodnotit jednotlivé typy sítí. Škály jsou
 samozřejmě relativní.

 \begin{enumerate}
 	\item Cena zavedení (počet kabelů a složitost celé topologie)
 	\item Spolehlivost (hledání chybných spojení v síti - nadbytečné spoje)
 	\item Složitost zvětšování (přidávání nových zařízení do sítě)
 	\item Zabezpečení (některé sítě mohou být fundamentálně lepší na zabezpečení
 	 než ostatní)
 \end{enumerate} 
\end{frame}


\section{Jednotlivé topologie}
\label{sec:jednotlive-topologie}


\subsection{Mash}
\label{sec:mash}

\begin{frame}{Co je \textbf{MASH} topologie?}
 Mash toplogie znamená, že propojíme každý počítač s hodně ostatními. Jestliže
 propojíme každý s každým jedná se o \emph{full Mash}. V tom případě máme
$\binom{N}{2}$ spojení. Počítače, i když sami nic neposílají, tak často musí být
zapojené aby mohli přeposílat packety které se přes ně posílají.

 \includegraphics[scale=0.2]{mash.png}

\end{frame}
\begin{frame}{Porovnání}

\begin{columns}
    \begin{column}{0.45\textwidth}
      \colheader{\textbf{Výhody}}
        \begin{itemize}
            \item Hodně nadbytečných spojů - odolné proti výpadkům.
            \item Žádný Single Point of Failure (SPF). 
            \item Málokdy je zahlcená.
            \item Přidání nového zařízení neomezí ty již v síti.
        \end{itemize}
    \end{column}
    \begin{column}{0.45\textwidth}  %%<--- here
        \colheader{\textbf{Nevýhody}}
        \begin{itemize}
            \item Velký počet kabelů stojí hodně peněz.
            \item Údržba těchto sítí je složitá
            \item Přidání jednoho stroje je pro ten stroj jako takový drahé a
             pomalé.
            \item Počítače běží více než by museli
        \end{itemize}
    \end{column}
    \end{columns}
    \vspace{10pt}

\textbf{Využití:} Používá se když si nemůžeme dovolit výpadky i za větší
cenu: nemocnice, letadla, vojenství atd. I když nebudeme mít full mash tak cena
s počtem roste rychle proto jich musíme mít co nejméně.
 
\end{frame}


\subsection{Ring}
\label{sec:ring}

\begin{frame}{Co je \textbf{RING} topologie?}
 Topologie prstenu spojuje zařízení jak byste čekali. Síť je spojená tak, že
 tvoří jednu velkou kružnici a data v ní cestují jedním nebo oběma směry. Protože
 data pořád krouží po síti, musí být všechny potítače zapnuté aby síť fungovala.

 \includegraphics[scale=0.25]{ring.jpg}
 
\end{frame}

\begin{frame}{Porovnání}

\begin{columns}
    \begin{column}{0.45\textwidth}
      \colheader{\textbf{Výhody}}
        \begin{itemize}
            \item Minimalizovaná packet kolize díky jasným směrům.
            \item Díky malému počtu spojení je levná.
            \item Nepotřebuje žádný server který by to řídil.
            \item Pro jednotlivce je jednoduché se připojit
        \end{itemize}
    \end{column}
    \begin{column}{0.45\textwidth}  %%<--- here
        \colheader{\textbf{Nevýhody}}
        \begin{itemize}
            \item Při přidávání nového zařízení celá síť vypadne
            \item Všechny počítače musí být neustále zapnuté aby síť fungovala.
            \item Každý stroj je SPF.
            \item Můj packet i když šifrovaný si mohou přečíst všichni v síti.
        \end{itemize}
    \end{column}
    \end{columns}
    \vspace{10pt}

\textbf{Využití:} v industriálních systémech které kontrolují chod celé fabriky.
Dále v malých lokálních sítích, které volí nižší cenu na úkor 100\% spolehlivosti.

\end{frame}

\subsection{Bus}
\label{sec:bus}
\begin{frame}{Co je \textbf{BUS} topologie?}
 Sběrnicová topologie funguje na principu jedné sběrnice (bus-line) do které
 jsou všechny přístroje zapojené. Po této lince se spolu všechny zařízení baví.

 \includegraphics[scale=0.9]{bus.png}
\end{frame}

\begin{frame}{Porovnání}

\begin{columns}
    \begin{column}{0.45\textwidth}
      \colheader{\textbf{Výhody}}
        \begin{itemize}
            \item Pro malé sítě se jedná o velmi levnou a jednoduchou
             alternativu.
            \item Velmi rychlý přenos dat.
            \item Jednoduché přidávání nových zařízení za běhu.
            \item Jednoduše provozovatelná.
        \end{itemize}
    \end{column}
    \begin{column}{0.45\textwidth}  %%<--- here
        \colheader{\textbf{Nevýhody}}
        \begin{itemize}
            \item Pro větší počty zařízení se už nevyplatí a také klesá
             rychlost.
            \item Bus-line je SPF.
            \item Kvůli jednomu kabelu a více přístrojům co přes něj můžou
             posílat je vysoký packet loss.
        \end{itemize}
    \end{column}
    \end{columns}
    \vspace{10pt}

\textbf{Využití:} hlavně v menších sítích kde jde hlavně o co nejmenší cenu a
jednoduché zacházení a nevadí větší packet loss. Například sítě ve škole nebo
doma.

\end{frame}


\subsection{Star}
\label{sec:star}
\begin{frame}{Co je to \textbf{STAR} topologie?}
 Hvězdicová topologie funguje tak, že má všechny přístroje napojené na centrální
 server přes který tedy jdou všechny packety. Tento server je buď aktivní nebo
 pasivní.

 Aktivní může například ještě šifrovat data nebo na nich sám provádět nějakou
 akci případně je zapisovat do databáze atd. Ten pasivní nedělá nic a jen je
 přeposílá tam kam jsou určena.

 \includegraphics[scale=0.25]{star.png}
\end{frame}

\begin{frame}{Porovnání}

\begin{columns}
    \begin{column}{0.45\textwidth}
      \colheader{\textbf{Výhody}}
        \begin{itemize}
         \item Pro velké sítě je efektivní.
         \item Velmi snadné přidávání/odebírání zařízení.
         \item Jednoduchá detekce chybného spojení a malá kolize packetů.

        \end{itemize}
    \end{column}
    \begin{column}{0.45\textwidth}  %%<--- here
        \colheader{\textbf{Nevýhody}}
        \begin{itemize}
            \item  Server je SPF. 
            \item Severy bývají velmi drahé a náročné na provozování.
            \item Jestliže si chtějí přístroje posílat hlavně věci mezi sebou,
             server celou akci notně zpomaluje.

        \end{itemize}
    \end{column}
    \end{columns}
    \vspace{10pt}

\textbf{Využití:} například v bankovnictví. Díky centrálnímu hubu je zařízená
bezpečnost a hlavně je jednoduché přidávat nové bankomaty. 
\end{frame}


\subsection{Tree}
\label{sec:tree}

\begin{frame}{Co je \textbf{TREE} topologie?}
    Stromová topologie funguje tak že jeden server je braný jako kořen a ostatní
    jsou zapojeny jakoby do stromu. To jaký má každý vrchol stupeň se může lišit,
    ale malý je neefektivní a velký vyústí v časté přetížení.

    \includegraphics[scale=0.4]{tree.png}
\end{frame}

\begin{frame}{Porovnání}

\begin{columns}
    \begin{column}{0.45\textwidth}
      \colheader{\textbf{Výhody}}
        \begin{itemize}
         \item Velmi jednoduché na přidávání nových zařízení a velmi modulární.
         \item Výpadek serveru na nižší hladině neohrozí nic kromě svého
             podstromu.
         \item Servery se mohou starat o šifrování.

        \end{itemize}
    \end{column}
    \begin{column}{0.45\textwidth}  %%<--- here
        \colheader{\textbf{Nevýhody}}
        \begin{itemize}
            \item  Root server je SPF.
            \item Je zapotřebí hodně serverů a to se prodraží.
        \end{itemize}
    \end{column}
    \end{columns}
    \vspace{10pt}

\textbf{Využití:} Funguje na tom vlastně celý internet (hierarchie v doménách
pomocí lomítek). Obecně na velkých sítích kde si většinou povídají jen skupiny u
sebe.

\end{frame}

\section{Závěr}
\label{sec:zaver}

\begin{frame}{Co si odnést?}
 \begin{itemize}
  \item Sítí je tuna a musíme si vybrat tu co nejlíp sedí na to co chceme
      (nemůžeme mít všechno).
  \item Na menší sítě se vyplatí \textbf{bus} nebo \textbf{mash}.
  \item Na ty větší \textbf{start},\textbf{ring} nebo \textbf{tree}.
 \end{itemize}
 
\end{frame}

\end{document}

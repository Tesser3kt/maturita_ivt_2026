\documentclass[aspectratio=169,xcolor=dvipsnames, t]{beamer}
\usepackage{fontspec} % Allows using custom font. MUST be before loading the theme!
\usetheme{SimplePlusAIC}
\usepackage{hyperref}
\usepackage{graphicx} % Allows including images
\usepackage{booktabs} % Allows the use of \toprule, \midrule and  \bottomrule in tables
\usepackage{svg} %allows using svg figures
\usepackage{tikz}
\usepackage{makecell}
\usepackage{wrapfig}
\usepackage[czech]{babel}
% ADD YOUR PACKAGES BELOW

%----------------------------------------------------------------------------------------
%	TITLE PAGE CONFIGURATION
%----------------------------------------------------------------------------------------

\title[]{File system} % The short title appears at the bottom of every slide, the full title is only on the title page
\subtitle{Maturitní seminář z IVT}

\author[Dusart]{Eric Dusart}
\institute[GEVO]{Gymnázium Evolution Jižní Město}
\date{\today}
\begin{document}

\maketitlepage
{
\setbeamertemplate{background}
{
    \includegraphics[width=\paperwidth,height=\paperheight]{AICStyleData/logos/mene_polygonu_bg.png}
}

\begin{frame}[t]{Obsah}
    \tableofcontents
\end{frame}
}
%------------------------------------------------
% Section divider frame
\section{Hlavní info}

{
\setbeamertemplate{background}
{
    \includegraphics[width=\paperwidth,height=\paperheight]{AICStyleData/logos/mene_polygonu_bg.png}
}
\begin{frame}{Hlavní informace}
    \textbf{\large K čemu to je? Proč to potřebujeme? \dots ?}
        \begin{itemize}
            \item Spravuje organizaci souborů.
            \item Hierarchie všech složek a souborů.
            \item Unifikuje přístup k datům všem aplikacím / procesům.
            \begin{itemize}
                \item Jinak by aplikace nekompatibilně využívaly data $\rightarrow$ korupce dat.
            \end{itemize}
            \item Existuje mnoho typů vytvořených pro různé typy externích pamětí.
            \item Virtuální - \emph{tmpfs}
        \end{itemize}
\end{frame}
}
{
\setbeamertemplate{background}
{
    \includegraphics[width=\paperwidth,height=\paperheight]{AICStyleData/logos/mene_polygonu_bg.png}
}
\section{Typy file systémů}
\begin{frame}{Typy file systémů}
\begin{itemize}
    \item Transactional file systems.
    \begin{itemize}
    \item Journaling file systems (JFS).
    \item Buď všechny změny provedeny, nebo žádná.
    \item Logují změny, které se mají provést na disku.
    \end{itemize}
    \item Versioning file systems.
    \begin{itemize}
    \item Vytvářejí backupy souborů a systému. Umožňují obnovu do předchozích verzí.
    \end{itemize}
\end{itemize}
\end{frame}
\begin{frame}{Typy file systémů}
    \begin{columns}
    \begin{column}{0.45\textwidth}
      \colheader{EXT4}
        \begin{itemize}
            \item Linux.
            \item Maximální velikost souboru: 16TB (4KB bloky).
            \item Maximální velikost file systému: 1EB (64KB bloky).
        \end{itemize}
    \end{column}
    \begin{column}{0.45\textwidth}  %%<--- here
        \colheader{NTFS}
        \begin{itemize}
            \item Windows.
            \item Některé systémy mají omezenou podporu.
        \end{itemize}
    \end{column}
    \end{columns}
\end{frame}
\begin{frame}{Typy file systémů}
    \begin{columns}
    \begin{column}{0.45\textwidth}
      \colheader{BTRFS}
        \begin{itemize}
            \item Podporuje snapshoty.
            \item Poměrně nový (2009).
            \item Copy-on-write (COW).
        \end{itemize}
    \end{column}
    \begin{column}{0.45\textwidth}  %%<--- here
        \colheader{HFS / APFS}
        \begin{itemize}
            \item Apple.
            \item HFS (1985), APFS (2017).
            \item APFS optimalizovaný pro SSD.
            \item APFS Podporuje snapshoty.
        \end{itemize}
    \end{column}
    \end{columns}
\end{frame}
\begin{frame}{Typy file systémů}
    \begin{columns}
    \begin{column}{0.45\textwidth}
      \colheader{FAT-32}
        \begin{itemize}
            \item File Allocation Table.
            \item Velmi jednoduchý, podporovaný skoro všude.
            \item Maximální velikost souboru: 4GB.

        \end{itemize}
    \end{column}
    \begin{column}{0.45\textwidth}  %%<--- here
        \colheader{exFAT}
        \begin{itemize}
            \item Extended File Allocation Table.
            \item Soubory až asi 16 EiB.
            \item Microsoft si ho patentoval, tím se jeho omezilo jeho využití.
            \item Používá se v SDXC kartách nebo flashkách.
            \item Datum a čas: (1980 - 2107).
            \item Nepodporuje kompresi ani šifrování.
        \end{itemize}
    \end{column}
    \end{columns}
\end{frame}
\begin{frame}{Typy file systémů}
    \begin{columns}
    \begin{column}{0.45\textwidth}
      \colheader{ZFS}
        \begin{itemize}
            \item Zettabyte File System.
            \item Maximální velikost file systému okolo 256 trilliónů yobibytů.
        \end{itemize}
    \end{column}
    \begin{column}{0.45\textwidth}  %%<--- here
        \colheader{tmpfs}
        \begin{itemize}
            \item Data jsou uloženy v RAM.
            \item Velmi rychlý.
            \item Data jsou ztracena po restartu.
            \item /tmp
        \end{itemize}
    \end{column}
    \end{columns}
\end{frame}
}
{
\setbeamertemplate{background}
{
    \includegraphics[width=\paperwidth,height=\paperheight]{AICStyleData/logos/mene_polygonu_bg.png}
}
\section{Fungování file systémů}
\begin{frame}{Fungování file systémů - Architektura}
Fungování file systému lze rozdělit do tří \uv{vrstev}:
\begin{itemize}
    \item \uv{Logická} vrstva.
    \begin{itemize}
        \item Poskytuje přístup k souborům, operace se soubory (open, read, write, \dots).
        \item Dává instrukce nižším vrstvám.
    \end{itemize}
    \item Virtuální vrstva (optional).
    \item Fyzická vrstva - low level přístup k externí paměti. Například psaní bloků do harddisku. Používá drivery zařízení (paměti).
\end{itemize}
\end{frame}
}
{
\setbeamertemplate{background}
{
    \includegraphics[width=\paperwidth,height=\paperheight]{AICStyleData/logos/mene_polygonu_bg.png}
}
\section{Organizace dat na disku}
\begin{frame}{Organizace dat na disku}
\begin{itemize}
    \item Pevný disk bývá rozdělen na oddíly (partitions).
    \item Každý partition má alespoň jeden file system.
\end{itemize}
Informace ve file systému se dělí na:
\begin{itemize}
    \item Metadata
    \begin{itemize}
        \item Data, která poskytují informace o jiných datech.
        \item Například velikost souboru, datum vytvoření, atd\dots
        \item Různé typy file systémů mají různý počet metadat.
    \end{itemize}
    \item Data
    \begin{itemize}
        \item Obsah souboru, který chceme přečíst.
    \end{itemize}
\end{itemize}

\end{frame}
}

{
\setbeamertemplate{background}
{
    \includegraphics[width=\paperwidth,height=\paperheight]{AICStyleData/logos/mene_polygonu_bg.png}
}
\section{Fragmentace a defragmentace}
\begin{frame}{Fragmentace a defragmentace}
\textbf{\large Co to je?}
\begin{itemize}
\item Stává se, když soubory nejsou uloženy v paměti po sobě.
\item Zpomaluje přístup k souborům.
\item Defragmentace - proces, kdy se soubory přesouvají tak, aby byly uloženy po sobě.
\item Gif: \href{https://upload.wikimedia.org/wikipedia/commons/d/d0/FragmentationDefragmentation.gif}{\textcolor{orange}{Defragmentace}} 
\end{itemize}

\end{frame}
}





\finalpagetext{Děkuji za pozornost}
%----------------------------------------------------------------------------------------
\makefinalpage
%----------------------------------------------------------------------------------------
\end{document}

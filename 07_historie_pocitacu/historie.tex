\documentclass{beamer}


\usetheme{Madrid}
\usepackage{graphicx} % Allows including images
\usepackage{booktabs} % Allows the use of \toprule, \midrule and \bottomrule in tables
\usepackage[czech]{babel}
%----------------------------------------------------------------------------------------
%	TITLE PAGE
%----------------------------------------------------------------------------------------

\title[amogus]{Historie počítačů} % The short title appears at the bottom of every slide, the full title is only on the title page

\author{Jáchym Löwenhöffer} % Your name
\institute[GEVO] % Your institution as it will appear on the bottom of every slide, may be shorthand to save space
{
Gynekologická Evaluace Velkých Obrazů \\ % Your institution for the title page
\medskip
\textit{jachym.lowenhoffer@gmail.com} % Your email address
}
\date{\today} % Date, can be changed to a custom date

\graphicspath{{./pics/}}

\AtBeginSection[]
{
  \begin{frame}
  \vfill
  \centering
  \begin{beamercolorbox}[sep=8pt,center,shadow=true,rounded=true]{title}
    \usebeamerfont{title}\insertsectionhead\par%
  \end{beamercolorbox}
  \vfill
  \end{frame}
}

\begin{document}

\begin{frame}
	\titlepage % Print the title page as the first slide
\end{frame}

\begin{frame}
	\frametitle{Přehled} % Table of contents slide, comment this block out to remove it
	\tableofcontents % Throughout your presentation, if you choose to use \section{} and \subsection{} commands, these will automatically be printed on this slide as an overview of your presentation
\end{frame}

%----------------------------------------------------------------------------------------
%	PRESENTATION SLIDES
%----------------------------------------------------------------------------------------

\begin{frame}
	\frametitle{Komp}
Abychom se mohli bavit o tom co je počítač musíme si ho definovat.
  \begin{block}{Definice počítače}
	 Provádí výpočty na určitých datech. Historicky nemusel být programovatelný,
	 ale teď již je tak často definovaný.
	\end{block}
	Podstata počítačů je aby lidem pomáhali se složitými výpočty. Tuto premisu
	plnili v minulosti například logaritmické tabulky, ovšem ty bychom z dnešního
	pohledu za počítače považovat nemohli.
\end{frame}

\begin{frame}
 \frametitle{Turingův stroj}
 Krom počítačů fyzických máme i počítačové modely, které slouží hlavně jako
 teoretické koncepty.

 \centering
 \includegraphics[scale=0.3]{Turing-machine.png}

\raggedright

 Turingův stroj je teoretický model s nekonečnou pamětí a jednoduchou hlavou,
 která buď čte, píše nebo se hýbe. To vše podle instrukcí které má.

 Podle tohoto stroje budeme posuzovat fyzické modely počítačů.
\end{frame}

\section{Předchůdci počítačů}
\label{sec:fde-cycle}

\begin{frame}
 \frametitle{Analogové a číslicové}
 \begin{block}{Analogové}
  Vstup berou veskrze spojitě a fungují na základě mechanických principů. Nejsou
  programovatelné a dají se tedy použít jen na velmi specifické a jednoduché
  úkoly.
 \end{block}

 \begin{block}{Číslicové}
  Tyto počítače jsou v principu ty co známe. Oproti analogovým berou vstup jako
  diskrétní sekvenci nul a jedniček (nebo jiných číslic počítáme-li úlety ze
  začátku počítačů). Fungují na fyzikálních principech a logických bránách.
  
 \end{block}

Historicky první byly ty analogové.
 \end{frame}

\begin{frame}
\frametitle{Byl Turing Turingovsky kompletní?}
Abychom mohli nějak klasifikovat programovatelnost počítače používáme takzvaný
Turingův test.
\begin{block}{Turignovsky úplný}
nazveme instrukční sadu jestliže je schopna simulovat každou další Turingovsky
úplnou sadu. Někdy se také definuje jako schopnost instrukční sady simulovat
Turingův stroj. 
\end{block}
Prvním návrh na Turingovsky kompletní počítač měl Ch. Babbage v roce 1833. Tento
projekt nikdy nedošel zdárného konce.
\end{frame}

\section{Ve válce múzy mlčí}
\begin{frame}
 \frametitle{Ale vědci ne, get fucked kalkulačky jsou tady!!}
Cílem vývoje za druhé světové války bylo získat díky větší výpočetní síle
převahu. Němci i Američani měli vlastní prototypy, které byli velmi složité ale
již Turingovsky kompletní a občas i použitelné.
\vfill
Input se stále musel zadávat ručně a jejich programování fungovalo přes děrné
štítky. Rychlost výpočetních instrukcí byla horší než jedna za sekundu.
\vfill
Tyto začátky počítačů ukazují jak je lidstvo ochotné důvěřovat technologii,
která ze začátku ničemu nepomáhá a svět ve kterém žijeme ukazuje jak se to čas
od času může vyplatit.
\end{frame}

\begin{frame}
 \frametitle{Omg rezistory}
\end{frame}
\begin{frame}
 \frametitle{Omg polovodiče}
\end{frame}
\begin{frame}
 \frametitle{Je libo počítač do kapsy?}
\end{frame}

\section{Budoucnost}
\begin{frame}
 \frametitle{Kvantové počítače.}
\end{frame}
\end{document}
